\documentclass[%
    %draft,
    %submission,
    %compressed,
    final,
    %
    %technote,
    %internal,
    %submitted,
    %inpress,
    reprint,
    %
    %titlepage,
    notitlepage,
    %anonymous,
    narroweqnarray,
    inline,
    twoside,
    invited
    ]{ieee}

\usepackage[utf8]{inputenc}
\usepackage[spanish]{babel}
\usepackage{graphicx}
\usepackage{verbatim}
\usepackage{moreverb}
\usepackage{amsmath}
\usepackage{amsfonts}
\usepackage{amssymb}
\usepackage{fancybox}
\usepackage{float}
\usepackage{fancyvrb}
\usepackage{subfigure}
\usepackage{multirow}

\newcommand{\latexiie}{\LaTeX2{\Large$_\varepsilon$}}

\title{SIA: Trabajo pr\'atico 1\\ M\'etodos de b\'usqueda informados y no informados}
\author{Agustin Marseillan - 50134,
\and Federico Ramundo - 51596,
\and Conrado Mader Blanco - 51270}
\date{Marzo 2013}

\usepackage{natbib}
\usepackage{graphicx}

\begin{document}
\journal{Cátedra\ \ Sist.\ de\ Inteligencia\ Artificial,\ ITBA\ }

\maketitle
\begin{abstract}
El presente informe busca analizar y comparar distintas estrategias de b\'usqueda de la soluci\'on de un problema en particular (DeepTrip) haciendo uso de un motor de inferencias, como as\'i tambi\'en evaluar las mejoras obtenidas
mediante la aplicaci\'on de heur\'sticas.
\end{abstract}
\begin{keywords}
DeepTrip, A*, Greedy, DFS, BFS, IDDFS, General Problem Solver, search strategy
\end{keywords}
\section{Introducci\'on}
\PARstart El juego DeepTrip consta de un tablero de diez filas por ocho columnas, en el que cada celda puede tener una ficha de color (rojo, naranja, azul, verde, violeta, amarillo) o estar vac\'io. \\
\par El objetivo es agrupar tres o mas fichas del mismo color, haciendo asi que exploten y ganando puntos. Cuando un grupo de fichas explota, en su lugar pasa a haber vac\'io. Se dice que un par de fichas est\'an agrupadas si est\'an una arriba de la otra, o una al lado de la otra.\\
\par El juego cuenta con gravedad invertida, lo que quiere decir que las fichas "flotan": si hay un espacio vac\'io arriba de una ficha entonces \'esta se mover\'a hacia esa direcci\'on hasta encontrar otra ficha o el fin del tablero. 
Esta propiedad permite que puedan ocurrir varias explosiones de fichas encadenadas.\\
\par Para poder agrupar las fichas, el jugador debe mover las filas, corriendo todas las fichas en ella a la vez, haciendo que cuando una ficha se salga del tablero por un lado, aparezca por el otro.\\
\par En el juego original, se cuenta con un tiempo determinado para hacer la mayor cantidad de puntos posibles, y cada vez que se explotan fichas, adem\'as de ganar puntos, se gana tiempo extra para seguir jugando. A su vez, cuando una fila se vac\'ia completamente, se genera una nueva fila llena en la parte superior del tablero, haciendo que \'este nunca tenga filas vac\'ias. Para este trabajo, dado que el juego no cuenta con un estado de \textit{goal}, se simplific\'o la l\'ogica original eliminando la regeneraci\'on de filas, sacando el tiempo y los puntos, y estableciendo el \textit{goal} como el tablero en el cual no quedan fichas (se explotaron todas). Tambi\'en se experiment\'o con tableros de tama\~nos diversos, y con la cantidad de dististos colores de las fichas.
\section{Estados del problema}
\subsection{Estado inicial}
\par El estado inicial es un tablero lleno en el cual no hay tres o mas fichas adyacentes del mismo color. Para que el tablero sea v\'alido debe existir una soluci\'on, es decir, una combinaci\'on de movimientos que lleve a vaciar el tablero.
\subsection{Estado final}
\par El estado final es un tablero sin fichas, es decir en el cual todas sus celdas est\'an vacias.
\subsection{Estado intermedio}
\par Un estado intermedio es todo aquel que no es ni inicial ni final, es decir, tiene algunas fichas y algunas celdas vac\'ias.
\section{Modelado del problema}
\par El tablero de juego se model\'o en la clase Board como una matriz de enteros, en el cual cada n\'umero representa un color distinto, siendo el cero el vac\'io. Para agilizar cuentas, adem\'as esta clase cuenta con informaci\'on extra, como la cantidad de fichas que quedan de cada color o la cantidad de movimientos que se realizaron desde el tablero inicial hasta llegar al actual.
\section{Reglas}
\par Dado un tablero de $n x m$ el conjunto de reglas posibles son:\\

\emph{Para la fila $n$, correr $m$ cantidad de fichas, siendo $m$ un numero mayor a cero.}\\
\par El n\'umero $m$ debe ser mayor a cero dado que de ser igual, no se esta\'ia realizando movimiento alguno, y menor a la cantidad de columnas del tablero, ya que hacer m\'as ser\'ia repetir movimientos. Correr una fila $i$ posiciones significa:\\

\emph{Por cada ficha o vac\'io en la posici\'on $(x, y)$ pasar\'a a estar en la posici\'on $(x, (y+i)\%m)$ siendo $m$ la cantidad de columnas del tablero e $i$ la cantidad de posiciones a correr.}\\

\par Las reglas son aplicables s\'olo si la fila a la que afecta no esta vac\'ia, dado que no se estar\'ian moviendo fichas. \\
\par Tal y como estan definidas las reglas, por cada tablero puede llegar a haber $n*(m-1)$ ramificaciones, lo que provoca que el \'arbol de b\'usqueda sea muy ancho para tableros muy grandes. Por ello se decidi\'o trabajar con tableros de hasta $8x8$.\\
\par Para ver un ejemplo de la aplicaci\'on de una regla ver el apartado \ref{anexo:ejemplo} del anexo.

\section{Costos}
\par Dada la naturaleza del problema, la aplicaci\'on de todas las reglas tiene el mismo costo, siendo unitario. Esto se traduce a que el costo de ir de un estado al otro es la cantidad de movimientos realizados para llegar a \'el.

\section{Heur\'isticas}
\par Por la propiedad de gravedad existente en el problema, es posible explotar en cadena varias fichas. Esto hace que sea extremadamente dif\'icil el aproximar, dado un estado, la distancia a la soluci\'on, y provoca que la b\'usqueda de una buena heur\'istica admisible tambi\'en lo sea.\\
\par La detecci\'on temprana de estados irresolubles es muy importante para la optimizaci\'on de la b\'usqueda. Para ello se cuenta con las siguiente condicion que un estado debe cumplir para saber que es solucionable, que es com\'un a todas las heur\'isticas implementadas:\\

\emph{Un estado es resoluble si y s\'olo si la cantidad de cada color en su tablero es distinto de uno o dos.}\\

Esto es f\'acilmente demostrable, dado que de no ser asi ya se sabe de antemano que es imposible de explotarlos (se necesitan al menos tres), provocando que el tablero no pueda ser vaciado.\\
\par Existe un control para prevenir el c\'alculo de un mismo tablero dos veces para evitar ciclos infinitos.\\
\par Para las heur\'isticas detlladas a continuaci\'on definimos las siguientes funciones:\\
\begin{itemize}
\item $Fichas(t) =$ cantidad de fichas restantes en el tablero.
\item $Dim(t) = n*m =$ dimensi\'on del tablero.
\item $Colores(t) =$ la cantidad de colores distintos restantes del tablero.
\item $Color(t, i) =$ la cantidad de fichas del color $i$ que quedan en el tablero.
\item $Islas(t) =$ la cantidad de grupos de dos fichas del mismo color.
\item $Movs(t) =$ la cantidad de movimientos realizados desde el estado inicial hasta el actual.
\end{itemize}
\subsection{Heuristica I - Cantidad de fichas}
La heur\'istica corresponde a la f\'ormula:\\

\begin{center}
\begin{math}
h_1(t) = Fichas(t)
\end{math}\\
\end{center}

\par \'Esta heu\'istica es claramente no admisible dado que asume que se necesitan tantos movimientos como fichas restantes, lo cual es falso dado que siempre se explotan en m\'as de una, y como ya se mencion\'o antes, es posible vaciar un tablero con s\'olo un movimiento, siendo en teste caso $h^*(t)=1 < h_1(t) = Fichas(t) = Dim(t)$.\\

\subsection{Heuristica II - Cantidad de explotaciones}
\'Esta heur\'istica se calcula mediante la f\'ormula:\\

\begin{center}
\begin{math}
h_2(t) = \sum\limits_{i=1}^k \lfloor Color(t, i) / 3 \rfloor
\end{math}\\
\end{center}

Siendo $k$ la cantidad de diferentes colores en el tablero.\\
\par \'Esta heur\'istica asume que se realiza una explotaci\'on de tres fichas por movimiento, lo cual nuevamente la hace no admisible, dado que es posible ganar el tablero en un solo movimiento con explotaciones en cadena, haciendo que $h^*(t)=1 < h_2(t)$. Esto es as\'i porque no es posible que $h_2(t) = 1$ cuando el tablero esta lleno, porque significar\'ia que \'este esta compuesto por un solo grupo de tres fichas del mismo color, lo que no pasa por como est\'a definido el estado inicial del juego.\\

\subsection{Heuristica III - Colores restantes}
El valor heur\'istico de un estado viene dado por el resultado de:\\

\begin{center}
\begin{math}
h_3(t) = Fichas(t)*0,6 + Colores(t)*Dim(t)*0.4
\end{math}\\
\end{center}

\par En \'este caso el valor heur\'istico corresponde a un promedio ponderado entre la cantidad de fichas restantes y los colores faltantes, multiplicando este \'ultimo valor adem\'as por las dimensiones del tablero para darle a\'un m\'as peso en el promedio. Esto es as\'i porque se considera mucho mejor un tablero al cual ya se le eliminaron todas las fichas de un mismo color a uno al que todav\'ia le quedan algunas, dotando a esta heur\'istica de la estrategia de eliminar colores lo m\'as r\'apido posible.\\
\par \'Esta heur\'istica tampoco es admisible dado que, como se dijo anteriormente, puede ser que el tablero se gane en un movimiento, pero $h_3(t)$ en un tablero completo siempre es mayor a uno dado que $Fichas(t)*0,6 = Dim(t)*0,6 = n*m*0,6 > 1$ salvo que $n=1 \and m=1$, lo que no tiene sentido, y como $Colores(t)*Dim(t)*0.4 >0$ llegamos a que $h^*(t)=1 < h_3(t)$.
\subsection{Heuristica IV}
Corresponde a la siguiente f\'ormula:\\

\begin{center}
\begin{math}
h_4=\frac{Fichas(t)}{2(Islas(t)+1)}
\end{math}\\
\end{center}

\par En esta heu\'istica le da mayor valor a los estados que tienen m\'as fichas agrupadas de a dos, asumiendo que resultar\'a m\'as f\'acil encontrar una tercera para hacerlas explotar.\\
\par No es una heur\'istica admisible, puede verse en un contraejemplo en el anexo, apartado \ref{anexo:ejemploh4}.

\subsection{Heuristica V}
La heur\'istica corresponde a la f\'ormula:\\

\begin{center}
\begin{math}
h_5=\left\{\begin{matrix}
             Dim(t) $si  $t$ es un tablero inicial$ \\
             (Movs(t)*(Dim(t)-Fichas(t)+1))^{-1} $sin\'o$
             \end{matrix}
   \right.
\end{math}\\
\end{center}

\par \'Esta heur\'istica favorece a los estados que eliminaron mayor cantidad de fichas en menos movimientos. Diferencia m\'as los estados en una etapa temprana, donde hay m\'as libertad de movimientos, pero no es tan eficaz en estados m\'as cercanos al final. La funci\'on heur\'istica es una funci\'on partida, dado que su valor ser\'ia infinito en el tablero inicial si no lo fuera, porque $Movs(t)=0$ si $t$ es inicial. A fines pr\'acticos no es un problema porque nunca se calcula el valor heur\'istico de el nodo ra\'iz, pero te\'oricamente es incorrecto que ese valor sea infinito.\\
\par Se trata de una heur\'istica admisible dado que su valor siempre es menor o igual a uno. Por m\'as de que lo sea, no resulta pr\'actica dado que eval\'ua a muchos estados con el mismo valor, haciendo que el \'arbol de b\'usqueda sea muy ancho, tardando en encontrar la soluci\'on.
\subsection{Heur\'isticas descartadas}
Se pens\'o en una heur\'istica admisible en el cual el valor correspond\'ia a la formula:\\

\begin{center}
\begin{math}
    h(t) = \frac{Fichas(t)}{Dim(t)}
\end{math}\\
\end{center}

\par Decimos que esta heur\'istica es admisible dado que es imposible que sobreestime a $h^*(t)$. Esto es as\'i porque, a lo sumo, cuando el tablero lleno, se puede resuelve con un s\'olo movimiento, es decir, costo 1. Esto lleva a que el valor heur\'istico tambi\'en sea 1, ya que $Fichas(t)$ correspondier\'ia a su valor m\'aximo (porque el tablero esta lleno) que equivale a la dimensi\'on del tablero, es decir $Dim(t)=n*m$. Ene ste caso $h(t) = h^*(t)$. En cualquier otro caso, cuando el tablero no est\'a lleno, $h(t)$ vale menos que 1, porque $Fichas(t) < Dim(t)$, lo que nos lleva a que $h(t) < h^*(t)$. De estas dos situaciones se saca que $h(t) \le h^*(t)$.\\
\section{Resultados}
Para ver la tabla de los promedios de los diferentes tiempos tardados en los diferentes m\'etodos de b\'usqueda ver laa secciones \ref{anexo:tabla} y \ref{anexo:tabla2} del anexo.
\subsection{Comparaci\'on de algoritmos}
\subsubsection{Algoritmos no informados}
\par Era de esperarse que el tiempo tardado en BFS crezca mucho cuando el tablero crece de tama\~no, dado que como se explic\'o antes, la cantidad de nodos abiertos por cada nodo est\'a en el orden de $n*(m-1)$, sin embargo encuentra la soluci\'on con la cantidad m\'inima de pasos, a diferencia de DFS que la encuentra r\'apido, pero en un nivel m\'as profundo en el \'arbol. IDDFS, toma lo mejor de cada algortimo, siendo un poquito m\'as lento que DFS, pero encontrando la soluci\'on con la cantidad m\'inima de pasos.\\
\par En cuanto a la cantidad de nodos expandidos, no sorprende el hecho de que DFS es el que menos nodos expande, seguido de BFS, y por \'ultimo IDDFS, dado que \'este debe iterar creado y destruyendo varios nodos iguales varias veces.\\
\par No se puede hablar de la conveniencia de un m\'etodo sobre otro ya que esto depende del tablero inicial y no se aplica ninguna estrategia de juego para la explosi\'on de nodos en el \'arbol, por lo tanto encontrar una soluci\'on de forma r\'apida va a depender de que la soluci\'on aparezca en los nodos que primero explota cada m\'etodo. Por la naturaleza del problema, cada tablero suele tener m\'as de una soluci\'on y los mismos suelen resolverse al aplicar una sucesi\'on de varias reglas. Es por esto que podemos ver que el algoritmo de b\'usqueda DFS (e IDDFS) es el que ha llegado a encontrar una soluci\'on de forma m\'as r\'apida, ya que aplica varias reglas a un mismo tablero, antes de probar con otro, y BFS va aplicando pocas
reglas a todos los tableros.\\

\subsubsection{Algoritmos informados}
\par Se puede observar que $Greedy$ es m\'as eficaz en tableros peque\~nos que $A^*$, mientras que \'este \'ultimo es un poco m\'as lento pero permite encontrar soluciones en tableros mucho m\'as grandes. Esto se da de esta forma dado que la diferencia entre $Greedy$ y $A^*$ es su funci\'on $f(n)$, (siendo $f(n)_greedy = h(n)$ y $f(n)_{A*} = g(n) + h(n)$), lo que hace que $Greedy$ ignore el costo del estado, es decir la cantidad de pasos que le tardo llegar hasta ahi y s\'olo tenga en cuenta la heur\'istica. Esto puede provocar, que si las heur\'isticas eval\'uan los estados con valores similares, si no tenemos en cuenta el costo, el \'arbol de b\'usqueda resulta muy ancho por lo que es estad\'isticamente m\'as lento al encontrar la soluci\'on que con uno m\'as angosto, como se da en $A^*$. Esto se ve claramente si se echa un vistazo a la tabla de cantidad de nodos expandidos.\\

\subsection{Comparaci\'on de heur\'isticas}
\par Llama la tenci\'on como los tiempos en las distintas heur\'isticas en el algorimto $Greedy$ son muy cercanos, mientras que en $A^*$, las heur\'isticas $h_4$ y $h_5$ son mucho m\'as lentas que las dem\'as. La diferencia entre $Greedy$ y $A^*$ es su funci\'on $f(n)$ como se explic\'o anteriormente, lo que implica que la diferencia est\'a en $g(n)$, la funci\'on de costo, siendo m\'as cercana a las heur\'isticas 1, 2, y 3 que a las 4 y 5.\\
\par Podemos concluir que las heur\'isticas 2 y 3 son las m\'as eficaces a la hora de encontrar soluciones, ya sea en tableros peque\~nos como grandes, dada la velocidad con la que llegan a la soluci\'on. Esto puede explicarse dado que son lasnheur\'isticas que se aproximan m\'as a la realidad: En el caso de $h_2$, es f\'acil darse cuenta emp\'iricamente que la mayoria de las fichas explotadas son por grupos de a tres fichas; en el caso de $h_3$, tiene mucho sentido tratar de eliminar color por color para lograr el objetivo con mayor precisi\'on.\\
\par Las heur\'isticas 4 y 5 son buenas para tableros peque\~nos pero resultan inservibles en los que son m\'as grandes, sobretodo en $A^*$.\\
\newpage
\section{Conclusi\'on}
%Un trabajo como este nos ayudara a cambiar el mundo y atrapar a todos los pokemons. Gotta catch em all.\\
Es notable la diferencia que existe a la hora de buscar una soluci\'on a un problema tan sencillo utilizando b\'usquedas informadas y no informadas. Particularmente, en este caso en el que el \'arbol de derivaci\'on es excepcionalmente ancho, el uso de b\'usquedas informadas ayuda mucho a la hora de expandir los nodos, y darles un peso a la hora de elegir el pr\'oximo a expandir hace que el trabajo resulte \'ordenes de magnitud m\'as r\'apido.\\
\par Resulta indispensable el calculo de estados ya visitados o estados sin soluci\'on sobretodo en algoritmos como DFS que van en profundidad, ya que de no ser as\'i podr\'ian encontrarse en un ciclo infinito corriendo la misma fila una y otra vez sin llegar a ning\'un lado.\\
\par En cuanto a las heur\'isticas, llama la atenci\'on c\'omo las m\'as intuitivas fueron las m\'as eficaces, d\'ando una idea de que un algoritmo que se aproxima a la intuici\'on humana es el que lleva a encontrar la soluci\'on m\'as rapidamente.
\newpage 
\section{Anexo}
\subsection{Ejemplo de aplicaci\'on de una regla}\label{anexo:ejemplo}
Teniendo el siguiente tablero de 3x3:
\begin{center}
\begin{tabular}{ l c r }
  5 & 2 & 2 \\
  4 & 4 & 1 \\
  3 & 1 & 4 \\
\end{tabular}
\end{center}
\par Si aplicamos la regla \emph{correr 1 posici\'on la fila 3} contaremos con el siguiente tablero: \\
\begin{center}
\begin{tabular}{ l c r }
  5 & 2 & 2 \\
  4 & 4 & 1 \\
  1 & 4 & 3 \\
\end{tabular}
\end{center}
\par Como ahora se tienen tres fichas del mismo color (mismo n\'umero en este caso), este grupo explota y las fichas debajo de estas flotan, generando el siguiente tablero:
\begin{center}
\begin{tabular}{ l c r }
  5 & 2 & 2 \\
  1 & 0 & 1 \\
  0 & 0 & 3 \\
\end{tabular}
\end{center}

\subsection{Contraejemplo de $h_4$}\label{anexo:ejemploh4}
Si se cuenta con el tablero siguiente:
\begin{center}
\begin{tabular}{ l c r }
  1 & 2 & 3 \\
  2 & 3 & 1 \\
  1 & 2 & 3 \\
\end{tabular}
\end{center}
Vemos que $h_4(t) = \frac{Fichas(t)}{2(Islas(t)+1)} = \frac{9}{2(0+1)} = 4.5$.
Pero si aplicamos la regla \emph{correr 1 posici\'on la fila 2} tenemos el tablero:
\begin{center}
\begin{tabular}{ l c r }
  1 & 2 & 3 \\
  1 & 2 & 3 \\
  1 & 2 & 3 \\
\end{tabular}
\end{center}
Es facil ver que explotan todas las fichas, llegando al \textit{goal}, y ganando en un movimiento, por lo que $h^*(t)=1 < 4.5 = h_4(t) \Rightarrow h_4(t)$ no es admisible. 

\subsection{Tablas de resultados de tiempos}\label{anexo:tabla}
Las siguientes tablas muestran el tiempo promedio en milisegundos al resolver 20 tableros de cada tama\~no con cada algoritmo.
\subsubsection{No informados}
\begin{tabular}{l|c|c|c|c|c}
   & 4x4[ms] & 5x5[ms] & 6x6[ms] & 7x7[ms] & 8x8[ms] \\
  \hline
  BFS & 51 & 78 & $\infty$ & $\infty$ & $\infty$ \\
  DFS & 31 & 68 & 390 & 1002 & $\infty$ \\
  IDDFS & 12 & 47 & $\infty$ & $\infty$ & $\infty$ \\
\end{tabular}
\subsubsection{Greedy}
\begin{tabular}{l|c|c|c|c|c}
   & 4x4[ms] & 5x5[ms] & 6x6[ms] & 7x7[ms] & 8x8[ms] \\
  \hline
  $h_1$ & 7 & 92 & 154 & 545 & 377 \\
  $h_2$ & 1 & 101 & 203 & 415 & 371 \\
  $h_3$ & 1 & 72 & 366 & 534 & 361 \\
  $h_4$ & 11 & 15 & 155 & 304 & 263 \\
  $h_5$ & 2 & 5 & 479 & 1029 & 3678 \\
\end{tabular}
\subsubsection{$A^*$}
\begin{tabular}{l|c|c|c|c|c}
   & 4x4[ms] & 5x5[ms] & 6x6[ms] & 7x7[ms] & 8x8[ms] \\
  \hline
  $h_1$ & 6 & 253 & 201 & 355 & 113 \\
  $h_2$ & 3 & 4 & 123 & 299 & 431 \\
  $h_3$ & 1 & 8 & 131 & 128 & 36\\
  $h_4$ & 14 & 82 & $\infty$ & $\infty$ & $\infty$ \\
  $h_5$ & 15 & 68 & $\infty$ & $\infty$ & $\infty$ \\
\end{tabular}

\subsection{Tablas de resultados de cantidad de nodos}\label{anexo:tabla2}
Las siguientes tablas muestran la cantidad de nodos generado promedio al resolver 20 tableros de cada tama\~no con cada algoritmo.
\subsubsection{No informados}
\begin{tabular}{l|c|c|c|c|c}
   & 4x4 & 5x5 & 6x6 & 7x7 & 8x8 \\
  \hline
  BFS & 1245 & 5146 & $\infty$ & $\infty$ & $\infty$ \\
  DFS & 268 & 733 & 1388 & 2531 & $\infty$ \\
  IDDFS & 2135 & 7216 & $\infty$ & $\infty$ & $\infty$ \\
\end{tabular}
\subsubsection{Greedy}
\begin{tabular}{l|c|c|c|c|c}
   & 4x4 & 5x5 & 6x6 & 7x7 & 8x8 \\
  \hline
  $h_1$ & 213 & 788 & 1256 & 2178 & 1598 \\
  $h_2$ & 214 & 1270 & 3014 & 2987 & 2541 \\
  $h_3$ & 213 & 788 & 1452 & 2176 & 1598 \\
  $h_4$ & 285 & 296 & 1966 & 2863 & 1471 \\
  $h_5$ & 210 & 214 & 522 & 1313 & 4987 \\
\end{tabular}
\subsubsection{$A^*$}
\begin{tabular}{l|c|c|c|c|c}
   & 4x4 & 5x5 & 6x6 & 7x7 & 8x8 \\
  \hline
  $h_1$ & 206 & 446 & 895 & 1399 & 895 \\
  $h_2$ & 319 & 380 & 547 & 2721 & 1248 \\
  $h_3$ & 208 & 446 & 1772 & 1420 & 845 \\
  $h_4$ & 304 & 513 & $\infty$ & $\infty$ & $\infty$ \\
  $h_5$ & 254 & 544 & $\infty$ & $\infty$ & $\infty$ \\
\end{tabular}

%\begin{figure}[h!]
%\centering
%\includegraphics[scale=1.7]{universe.jpg}
%\caption{The Universe}
%\label{threadsVsSync}
%\end{figure}

\end{document}
